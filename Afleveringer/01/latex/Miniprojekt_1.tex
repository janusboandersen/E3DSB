

\documentclass[a4paper]{report}
\usepackage[margin=1in]{geometry}
\usepackage{polyglossia}
\setdefaultlanguage{danish}
\usepackage{graphicx}
\usepackage{xcolor}
\usepackage{amsfonts, amsmath, amssymb}
\usepackage{float} %ensures that figures / tables can be floated on page
\usepackage{hyperref}
\usepackage{fancyref}

\usepackage[numbered,framed]{matlab-prettifier}
\lstset{
  language           = Matlab,
  style              = Matlab-editor,
  basicstyle         = \mlttfamily,
  escapechar         = `,
  mlshowsectionrules = true
}

\usepackage[binary-units=true]{siunitx}
\sisetup{detect-all}

%\definecolor{vertmatlab}{RGB}{28,160,55}
%\definecolor{mauvematlab}{RGB}{155,71,239}
%\definecolor{fond}{RGB}{246,246,246}
\definecolor{lightgray}{gray}{0.5}

\sloppy
\setlength{\parindent}{0pt}

\usepackage{titlesec}
\titleformat{\chapter}{\huge\bf}{\thechapter.}{20pt}{\huge\bf}
\titleclass{\chapter}{straight}

\author{Janus Bo Andersen \thanks{ja67494@post.au.dk}}

\begin{document}

\pagenumbering{roman}


    
    
    \title{E3DSB miniprojekt 1 - Tidsdomæneanalyse}
        

    \maketitle

\tableofcontents
\newpage

\pagenumbering{arabic}

    \begin{par}

\chapter{Indledning}
Dette første miniprojekt i E3DSB behandler tre lydsignaler med analyser i tidsdomænet.
Opgaven er løst individuelt.
Dette dokument er genereret af Matlab med en XSL-template.
Matlab-kode og template findes på \url{https://github.com/janusboandersen/E3DSB}.
Følgende lydklip benyttes \\
\begin{table}[H]
\centering
\begin{tabular}{| c | c | c | c |} \hline
Signal & Skæring & Genre & Samplingsfrekv. \\ \hline
$s_1$ & Spit Out the Bone & Thrash-metal & 44.1 \si{\kilo\hertz} \\ \hline
$s_2$ & The Wayfaring Stranger & Bluegrass & 96 \si{\kilo\hertz} \\ \hline
$s_3$ & Svanesøen & Klassisk & 44.1 \si{\kilo\hertz} \\ \hline
\end{tabular}\caption{3 signaler behandlet i analysen}\label{tab:lydklip}\end{table}

\end{par} 
\begin{par}

\chapter{Analyser}
Før analyser ryddes der op i \texttt{Workspace}.\\

\end{par} 

\begin{lstlisting}[language=Matlab, style=Matlab-editor]
clc; clear all; close all;
\end{lstlisting}



\section{Afspilning}

        \begin{par}

Filen åbnes med \texttt{load}.
Signaler kan afspilles med \texttt{soundsc(signal, fs)}.
Samplingsfrekvensen $f_s$ sættes efter værdi i tabel~\ref{tab:lydklip}.

\end{par} 

\begin{lstlisting}[language=Matlab, style=Matlab-editor]
load('miniprojekt1_lydklip.mat')
soundsc(s1, fs_s1)
clear('sound');
\end{lstlisting}



\section{Bestemmelse af antal samples}

        


\section{Plot af signal}

        


\section{Min, max, RMS og energi}

        


\section{Venstre vs. højre kanal (for $s_1$)}

        


\section{Nedsampling af signal (for $s_1$)}

        


\section{Fade-out med envelopes (for $s_2$)}

        \begin{par}
\chapter{Konklusion}
\end{par} 



\end{document}

