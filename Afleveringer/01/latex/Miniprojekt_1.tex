

\documentclass[a4paper]{report}
\usepackage[margin=1in]{geometry}
\usepackage{polyglossia}
\setdefaultlanguage{danish}
\usepackage{graphicx}
\usepackage{xcolor}
\usepackage{amsfonts, amsmath, amssymb}
\usepackage{float} %ensures that figures / tables can be floated on page
\usepackage{hyperref}
\usepackage{fancyref}

\usepackage{setspace}
\onehalfspacing % line spacing (is equal to baselinestretch 1.33)

\usepackage[numbered,framed]{matlab-prettifier}
\lstset{
  language           = Matlab,
  style              = Matlab-editor,
  basicstyle         = \mlttfamily,
  escapechar         = `,
  mlshowsectionrules = true
}

\usepackage[binary-units=true]{siunitx}
\sisetup{detect-all}

%\definecolor{vertmatlab}{RGB}{28,160,55}
%\definecolor{mauvematlab}{RGB}{155,71,239}
%\definecolor{fond}{RGB}{246,246,246}
\definecolor{lightgray}{gray}{0.7}

\sloppy
\setlength{\parindent}{0pt}

\usepackage{titlesec}
\titleformat{\chapter}{\huge\bf}{\thechapter.}{20pt}{\huge\bf}
\titleclass{\chapter}{straight}

\author{Janus Bo Andersen \thanks{ja67494@post.au.dk}}

\begin{document}

\pagenumbering{roman}


    
    
    \title{E3DSB miniprojekt 1 - Tidsdomæneanalyse}
        

    \maketitle

\tableofcontents
\newpage

\pagenumbering{arabic}

    \begin{par}

\chapter{Indledning}
Dette første miniprojekt i E3DSB behandler tre lydsignaler med analyser i tidsdomænet.
Opgaven er løst individuelt.
Dette dokument er genereret i Matlab med en XSL-template.
Matlab-kode og template findes på \url{https://github.com/janusboandersen/E3DSB}.
Følgende lydklip benyttes \\
\begin{table}[H]
\centering
\begin{tabular}{| c | c | c | c |} \hline
Signal & Skæring & Genre & Samplingsfrekv. \\ \hline
$s_1$ & Spit Out the Bone & Thrash-metal & 44.1 \si{\kilo\hertz} \\ \hline
$s_2$ & The Wayfaring Stranger & Bluegrass & 96 \si{\kilo\hertz} \\ \hline
$s_3$ & Svanesøen & Klassisk & 44.1 \si{\kilo\hertz} \\ \hline
\end{tabular}\caption{3 signaler behandlet i analysen}\label{tab:lydklip}\end{table}

\end{par} 
\begin{par}

\chapter{Analyser}
Før analyser ryddes der op i \texttt{Workspace}.\\

\end{par} 

\begin{lstlisting}[language=Matlab, style=Matlab-editor]
clc; clear all; close all;
\end{lstlisting}



\section{Afspilning af lydklip}

        \begin{par}

Filen med signaler åbnes med \texttt{load}.
Signaler kan afspilles med \texttt{soundsc(signal, fs)}.
Samplingsfrekvensen $f_s$ sættes efter værdi i tabel~\ref{tab:lydklip}.
Samplingfrekvenser for de tre signaler er inkluderet i
\texttt{.mat}-filen.

\end{par} 

\begin{lstlisting}[language=Matlab, style=Matlab-editor]
load('miniprojekt1_lydklip.mat');   % åbn .mat-fil
soundsc(s1, fs_s1);                 % playback startes sådan her
clear('sound');                     % stop playback
\end{lstlisting}



\section{Bestemmelse af antal samples}

        \begin{par}

Et sample er en værdi, eller sæt af værdier, fra et givent punkt i tid.
Alle tre signaler er i stereo, så hver sample har to værdier.\\\\
Signalerne er repræsenteret som $N\times K$-matricer.
Antallet af rækker, $N$, repræsenterer antallet af samples.
$N$ kan findes med \texttt{length(matrix)}.
Antallet af søljer, $K$, er antallet af kanaler (værdier per sample).
Samlet antal af værdier i matricen er $NK$, antaget at ingen er \texttt{NaN}.\\

\end{par} 
\begin{par}

$N$ og $K$ kan bestemmes på en gang via \texttt{[N, K] = size(matrix)}.
Vi kan også bare benytte, at vi ved, at der er to kanaler, så $K = 2$. \\\\
Data samles i en tabel. Den kan udvides med signalernes afspilningstider.\\\\
Der er altså fx 1,323 millioner samples i signal $s_1$.
Signal $s_2$, som dog har højere samplingsfrekvens, har 2,5 gange flere
samples.
De tre lydklip har afspilningstider på mellem 30 og 35 sek.\\

\end{par} 

\begin{lstlisting}[language=Matlab, style=Matlab-editor]
signaler = {'s1'; 's2'; 's3'};
N = [length(s1); length(s2); length(s3)];           % antal samples
K = [2; 2; 2];                                      % antal kanaler
M = N.*K;                                           % antal værdier
samplingsfrek = [fs_s1; fs_s2; fs_s3];              % f_s fra .mat-fil
tid = N./samplingsfrek;                             % spilletid i sek.
T = table(signaler, N, K, M, samplingsfrek, tid)    % vis en datatabel
\end{lstlisting}

        \color{lightgray} \begin{verbatim}
T =

  3×6 table

    signaler        N         K        M         samplingsfrek    tid
    ________    __________    _    __________    _____________    ___

      's1'       1.323e+06    2     2.646e+06        44100        30 
      's2'        3.36e+06    2      6.72e+06        96000        35 
      's3'      1.4112e+06    2    2.8224e+06        44100        32 

\end{verbatim} \color{black}
    


\section{Plot af signal}

        \begin{par}

Når vi skal plotte signalerne med en tidsakse i sekunder, bruges det at
$t = n T_s = \frac{n}{f_s}$. Man bør plotte et diskret signal i et
stem-diagram, dvs. \texttt{stem}-funktionen, men for at få noget mindre
gnidret at se på, bruges \texttt{plot}. Til at danne akserne bruges
Matlabs \texttt{:}-operator.\\\\

\end{par} 

\begin{lstlisting}[language=Matlab, style=Matlab-editor]
t1 = [0:1:N(1)-1]'/fs_s1;                   % søjlevektor, dog ej vigtigt
t2 = [0:1:N(2)-1]'/fs_s2;
t3 = [0:1:N(3)-1]'/fs_s3;

% der gøres lidt arbejde for at få et rent latex layout
set(groot, 'defaultAxesTickLabelInterpreter','Latex');
set(groot, 'defaultLegendInterpreter','Latex');
set(groot, 'defaultTextInterpreter','Latex');

figure(1)                                   % figur med 3 stablede subplots
subplot(3,1,1);
plot(t1,s1);                                % signal 1
ylabel('$s_1$','Interpreter','Latex', 'FontSize', 15);
subplot(3,1,2);
plot(t2,s2);                                % signal 2
ylabel('$s_2$','Interpreter','Latex', 'FontSize', 15);
subplot(3,1,3);
plot(t3,s3);                                % signal 3
ylabel('$s_3$','Interpreter','Latex', 'FontSize', 15);
xlabel('$t [s]$','Interpreter','Latex', 'FontSize', 15);

% og en titel for hele diagrammet
sgtitle('Plot af $s_1$, $s_2$, $s_3$', 'Interpreter', 'Latex', 'FontSize', 20);
\end{lstlisting}

\begin{center}
    \includegraphics [width=5in]{Miniprojekt_1_01.eps}
\end{center}
\begin{par}

Plots viser ret tydeligt store forskelle i lydklippenes ``intensitet''.
Forstået på den måde, at lydklippet med thrash-metal har en gennemgående
høj amplitude (opleves som ``højt''), i modsætning til fx det klassiske stykke.
Nogle ville nok bare mene, at plottet over Metallicas nummer ligner ``støj'' :-).\\\\
Næste analyse kan måske give numeriske mål på disse visuelle observationer.\\

\end{par} 



\section{Min, max, energi og RMS}

        \begin{par}

I dette afsnit beregnes forskellige mål på signalernes lydmæssige ``karakter''.\\

\end{par} 
\begin{par}

\textbf{Overvejelser:}~
Signalerne er i stereo (2 kanaler / søjler).
Hvis vi har et system med to højttalere, giver det mening at betragte kanalerne separat (ikke sammenlagt).
Altså, jeg analyserer kanalerne i forlængelse, som en mono serie med $M=2N$ samples.
Denne løsning bruges, fordi det er sådan et menneske med to ører og sæt hovedtelefoner ville opleve signalet :-).
Det er også proportionalt til effekt og energiafsættelse i et system med to højttalere.\\\\
En sum eller et gennemsnit på tværs af kanalerne ville betyde, at kanaler
ude af fase kunne cancellere/eliminere hinanden.
Dette ville måske give mening som en simpel konvertering til mono, dvs. vi
kunne beregne mål på hvad der ville ske i et simpelt mono-system.\\

\end{par} 
\begin{par}

\textbf{Beregning:}~
Minimum og maksimum findes med hhv. \texttt{min()}~ og \texttt{max()}.
I tidsdomænet er effekten af et signal proportionalt til kvadratet på
amplituden. For en sekvens $x(n) \in \mathbb{R}$, $n = 0,\ldots,N-1$
defineres effekten som $x_{pwr}(n) = |x(n)|^2 = x(n)^2 $.
I diskret tid er energien i signalet summen af ``effekterne'', dvs.
$E_x = \sum_{n=0}^{N-1} |x(n)|^2 $.
Dette er også det indre produkt $\langle x(n), x(n) \rangle$.
RMS-værdien kan beregnes som kvadratroden af middeleffekten, dvs.
$x_{RMS} = \sqrt{\frac{1}{N}E_x}$.
Nu regnes alle serier så blot over $n = 0, \ldots, 2(N-1)$ jf. overvejelserne ovenfor.\\

\end{par} 

\begin{lstlisting}[language=Matlab, style=Matlab-editor]
s1_vec = reshape(s1,[],1);      % Reshape matricer til søjlevektorer:
s2_vec = reshape(s2,[],1);      % De har nu hver M = 2N rækker og 1 søjle
s3_vec = reshape(s3,[],1);      % N, M er selvfølgelig forskellige for hver

minima = [min(s1_vec); min(s2_vec); min(s3_vec)];
maxima = [max(s1_vec); max(s2_vec); max(s3_vec)];
energi = [sum(s1_vec.^2); sum(s2_vec.^2); sum(s3_vec.^2)];     % kvadratsum
rms = [energi(1)/M(1); energi(2)/M(2); energi(3)/M(3)].^(1/2); % kv.rod

T = table(signaler, N, M, minima, maxima, energi, rms)         % resultater
\end{lstlisting}

        \color{lightgray} \begin{verbatim}
T =

  3×7 table

    signaler        N             M          minima     maxima       energi        rms   
    ________    __________    __________    ________    _______    __________    ________

      's1'       1.323e+06     2.646e+06     -1.0166     1.0191    2.5336e+05     0.30944
      's2'        3.36e+06      6.72e+06    -0.61796    0.62791         34641    0.071797
      's3'      1.4112e+06    2.8224e+06    -0.85016    0.76907        5662.2     0.04479

\end{verbatim} \color{black}
    \begin{par}
Resultaterne (i tabellen) viser det, som plots også illustrerede: Der er mere energi i metal end i klassisk og bluegrass :-) Og højttalerne bliver varmere af at spille Metallica end af Tchaikovsky.
\end{par} 



\section{Venstre vs. højre kanal (for $s_1$)}

        \begin{par}

Man kan eksperimentere lidt for at finde ud af hvilken kanal, der er
højre, og hvilken der er venstre.
Man kan jo fx fylde den ene kanal med nuller, og så se, hvad der ``sker''.
Stereo bibeholdes ved at fastholde matricens $N\times K$-størrelse, men
med en kanal ``nullet''.\\

\end{par} 

\begin{lstlisting}[language=Matlab, style=Matlab-editor]
s1_left_stereo = s1;
s1_left_stereo(:,2) = zeros(N(1),1); % Nuller "højre" via 2. kanal
soundsc(s1_left_stereo, fs_s1);      % Bingo, det virkede
clear sound;

s1_right_stereo = s1;
s1_right_stereo(:,1) = zeros(N(1),1); % Nuller "venstre" via 1. kanal
soundsc(s1_right_stereo, fs_s1);
clear sound;
\end{lstlisting}
\begin{par}

Differensen mellem kanalerne kan også aflyttes.
Vi tager venstre minus højre.\\

\end{par} 

\begin{lstlisting}[language=Matlab, style=Matlab-editor]
s1_diff_mono = s1(:,1) - s1(:,2);       % venstre minus højre
soundsc(s1_diff_mono, fs_s1);
clear sound;
\end{lstlisting}
\begin{par}

Differensen mellem kanalerne giver en effekt af at lyden ``kommer'' et
bestemt sted fra, rumligt/spatialt (eller evt. at der er en genklang).
Fx vil en lille forsinkelse i den højre kanal snyde hjernen til at tro, at
lyden kom fra et sted tættere på det venstre øre.
Forsinkelse kan derfor benyttes til at ``flytte'' instrumenternes lyd i
rummet.\\\\
I dette lydklip oplever jeg, at alle instrumenter er tilstede i både
venstre og højre kanal, men i forskellig grad. Differensen afslører, at:
\begin{itemize}
\item Den hurtige lyd af J. Hetfields downpicking/strumming bevæger sig
mellem kanalerne.
\item Det gør lyden af L. Ulrichs lilletromme til dels også.
\item Det giver en fornemmelse af at være omringet af lyden.
\item Desuden er L. Ulrichs hi-hat placeret til venstre for midten på enkeltslagene,
men til højre for midten på triple-slaget.
\end{itemize}
Hvis klippet havde været længere, havde vi også tydeligt hørt den fede og
lidt mere melodiske del af guitarriffet (som starter ca. 40 sekunder
inde) placeret i venstre kanal.\\

\end{par} 



\section{Nedsampling af signal (for $s_1$)}

        \begin{par}

Der laves en nedsampling af signalet med en faktor 4.
Funktionen \texttt{resample(signal, fs\_ny, fs\_gl)}~ benyttes.\\

\end{par} 

\begin{lstlisting}[language=Matlab, style=Matlab-editor]
fs_s1_ny = fs_s1 / 4;                           % reduktion med faktor 4
s1_downsample = resample(s1, fs_s1_ny, fs_s1);  % downsampling
txt = sprintf("Nyt antal samples: %d", length(s1_downsample));
disp(txt);

soundsc(s1_downsample, fs_s1_ny);               % afspil nyt klip
clear sound;
\end{lstlisting}

        \color{lightgray} \begin{verbatim}Nyt antal samples: 330750
\end{verbatim} \color{black}
    \begin{par}
Det høres tydeligt, at downsampling har reduceret lydkvaliteten. Klippet lyder nu mere som internetradio i 90'erne, eller en dårlig YouTube-video.
\end{par} 



\section{Fade-out med envelopes (for $s_2$)}

        \begin{par}

Vi vil lave fade-out over den sidste tredjedel af signalet.
Dvs. cirka de sidste 12 af de i alt 35 sekunder.
Helt præcist skal indhyldningskurven påvirke de sidste 1,12 mio. samples.
Altså $N_{env,2} = \frac{1}{3} N_2 = 3360000/3 = 1120000$.
Der benyttes to forskellige metoder:
\begin{itemize}
\item Lineær envelope fra 100 til 5 pct.
\item Eksponentielt aftagende envelope fra 100 til 5 pct.
\end{itemize}
Metoden bliver at lave envelopes med den ønskede længde, og så applicere
dem på den sidste tredjedel af signalet.\\

\end{par} 
\begin{par}
\subsection{Lineær envelope}
\end{par} 
\begin{par}

Der skal over $N_{env,2}$~ samples foretages en \textbf{lineær} ``afskrivning''.
Funktionen er $f_{lin}(n) = -\alpha n$ for $n=0,\ldots,N_{env,2}-1$.
Yderpunkterne sættes $f_{lin}(0) = 1$ og $f_{lin}(N_{env,2}-1) = 0.05$.
Det giver en hældning på $ \alpha = -\frac{(0.05-1.00)}{N_{env,2}-1}=8.48\cdot 10^{-7}$.\\\\
Men det er naturligvis nemmere bare at bruge \texttt{linspace}...

\end{par} 

\begin{lstlisting}[language=Matlab, style=Matlab-editor]
N_env2 = N(2)/3;                            % antal samples der skal filtr.
lin_env2 = linspace(1.0, 0.05, N_env2)';    % lineær envelope
\end{lstlisting}
\begin{par}
\subsection{Eksponentiel envelope}
\end{par} 
\begin{par}

Der skal over $N_{env,2}$~ samples foretages en \textbf{eksponentiel} ``afskrivning''.
Funktionen er $g_{exp}(n) = \exp(-\gamma n)$ for $n=0,\ldots,N_{env,2}-1$.
Yderpunkterne sættes $g_{exp}(0) = 1$ og $g_{exp}(N_{env,2}-1) = 0.05$.
Det giver med lidt omskrivning en faktor på $ \gamma = -\frac{\ln(0.05)}{N_{env,2}-1}=2.67\cdot 10^{-6}$.\\\\

\end{par} 

\begin{lstlisting}[language=Matlab, style=Matlab-editor]
gamma = -log(0.05)/(N_env2 - 1);            % nb. ln = log()
exp_env2 = exp(-gamma*[0:N_env2-1])';       % vektoriseret exp envelope
\end{lstlisting}
\begin{par}
\subsection{Sammenligning af envelopes}
\end{par} 
\begin{par}

De to envelopes (indhyldningskurver) plottes, så vi kan se, om vi har
fået hvad vi ønskede...\\

\end{par} 

\begin{lstlisting}[language=Matlab, style=Matlab-editor]
figure(2)
subplot(2,1,1);
plot(lin_env2);
ylabel('$-\alpha n$','Interpreter','Latex', 'FontSize', 15);
dim1 = [.2 .45 .3 .3];                   % Placering af annotation
str_lin = '$\alpha = 8.48\cdot10^{-7}$';
annotation('textbox',dim1,'Interpreter', 'Latex', 'String',str_lin,'FitBoxToText','on', 'FontSize', 15);

subplot(2,1,2);
plot(exp_env2);
ylabel('$\exp(-\gamma n)$','Interpreter','Latex', 'FontSize', 15);
str_exp = '$\gamma = 2.67\cdot10^{-6}$';
dim2 = [.2 .13 .0 .3];                  % Placering af annotation
annotation('textbox',dim2,'Interpreter', 'Latex', 'String',str_exp,'FitBoxToText','on', 'FontSize', 15);

xlabel('$n$','Interpreter','Latex', 'FontSize', 15);
sgtitle('Sammenligning af envelopes', 'Interpreter', 'Latex', 'FontSize', 20);
\end{lstlisting}

\begin{center}
    \includegraphics [width=5in]{Miniprojekt_1_02.eps}
\end{center}
\begin{par}
Envelopes påtrykkes signalet direkte, selvom man nok også kunne have brugt \texttt{filter}-funktionen.
\end{par} 

\begin{lstlisting}[language=Matlab, style=Matlab-editor]
pad_ones = ones([2*N_env2, 1]);         % pad med 1-taller når sig. ej ændr
tot_lin_fade = [pad_ones; lin_env2];    % sammensæt for hele serien
tot_exp_fade = [pad_ones; exp_env2];

% Fade påtrykkes hver kanal
s2_lin_fade = s2;
s2_exp_fade = s2;
for k=1:2
    s2_lin_fade(:,k) = s2_lin_fade(:,k) .* tot_lin_fade;  % påtryk lineær
    s2_exp_fade(:,k) = s2_exp_fade(:,k) .* tot_exp_fade;  % påtryk eksp.
end

% Afspil resultaterne
soundsc(s2_lin_fade, fs_s2);
clear sound;

soundsc(s2_exp_fade, fs_s2);
clear sound;
\end{lstlisting}
\begin{par}
Det er nok smag og behag med de to forskellige typer. Jeg bryder mig bedst om den eksponentielle fade-out, fordi den hurtigere reducerer lydstyrken. Det mest naturlige ville nok være en logaritmisk fade, der matcher vores ørers og hjernes evne til at opfatte forskelle i lydniveauer, hvilket netop oftest er ``efter'' logaritimisk skala.
\end{par} 
\begin{par}
\chapter{Konklusion}
\end{par} 
\begin{par}

Dette miniprojekt har vist, hvordan man kan arbejde med digitale lydsignaler i
Matlab.\\\\
Det er interessant, hvordan relativt simple matematiske metoder kan
benyttes til at analysere og behandle digitale lydsignaler. Matlab gør
arbejdet nemt for os.

\end{par} 



\end{document}

