

\documentclass[a4paper]{report}
\usepackage[margin=1in]{geometry}
\usepackage{polyglossia}
\setdefaultlanguage{danish}
\usepackage{graphicx}
\usepackage{xcolor}
\usepackage{amsfonts, amsmath, amssymb}
\usepackage{float} %ensures that figures / tables can be floated on page
\usepackage{hyperref}
\usepackage{fancyref}

\usepackage{setspace}
\onehalfspacing % line spacing (is equal to baselinestretch 1.33)

\usepackage[numbered,framed]{matlab-prettifier}
\lstset{
  language           = Matlab,
  style              = Matlab-editor,
  basicstyle         = \mlttfamily,
  escapechar         = `,
  mlshowsectionrules = true
}

\usepackage[binary-units=true]{siunitx}
\sisetup{detect-all}

%\definecolor{vertmatlab}{RGB}{28,160,55}
%\definecolor{mauvematlab}{RGB}{155,71,239}
%\definecolor{fond}{RGB}{246,246,246}
\definecolor{lightgray}{gray}{0.7}

\sloppy
\setlength{\parindent}{0pt}

\usepackage{titlesec}
\titleformat{\chapter}{\huge\bf}{\thechapter.}{20pt}{\huge\bf}
\titleclass{\chapter}{straight}

\author{Janus Bo Andersen \thanks{ja67494@post.au.dk}}

\begin{document}

\pagenumbering{roman}


    
    
    \title{E3DSB miniprojekt 1 - Tidsdomæneanalyse}
        

    \maketitle

\tableofcontents
\newpage

\pagenumbering{arabic}

    \begin{par}

\chapter{Indledning}
Dette første miniprojekt i E3DSB behandler tre lydsignaler med analyser i tidsdomænet.
Opgaven er løst individuelt.
Dette dokument er genereret i Matlab med en XSL-template.
Matlab-kode og template findes på \url{https://github.com/janusboandersen/E3DSB}.
Følgende lydklip benyttes \\
\begin{table}[H]
\centering
\begin{tabular}{| c | c | c | c |} \hline
Signal & Skæring & Genre & Samplingsfrekv. \\ \hline
$s_1$ & Spit Out the Bone & Thrash-metal & 44.1 \si{\kilo\hertz} \\ \hline
$s_2$ & The Wayfaring Stranger & Bluegrass & 96 \si{\kilo\hertz} \\ \hline
$s_3$ & Svanesøen & Klassisk & 44.1 \si{\kilo\hertz} \\ \hline
\end{tabular}\caption{3 signaler behandlet i analysen}\label{tab:lydklip}\end{table}

\end{par} 
\begin{par}

\chapter{Analyser}
Før analyser ryddes der op i \texttt{Workspace}.\\

\end{par} 

\begin{lstlisting}[language=Matlab, style=Matlab-editor]
clc; clear('all'); close('all');
\end{lstlisting}



\section{Afspilning af lydklip}

        \begin{par}

Filen med signaler åbnes med \texttt{load}.
Signaler kan afspilles med \texttt{soundsc(signal, fs)}.
Samplingsfrekvensen $f_s$ sættes efter værdi i tabel~\ref{tab:lydklip}.
Samplingfrekvenser for de tre signaler er inkluderet i
\texttt{.mat}-filen.

\end{par} 

\begin{lstlisting}[language=Matlab, style=Matlab-editor]
load('miniprojekt1_lydklip.mat');
soundsc(s1, fs_s1); %playback
clear('sound'); %stop playback
\end{lstlisting}



\section{Bestemmelse af antal samples}

        


\section{Plot af signal}

        


\section{Min, max, RMS og energi}

        \begin{par}

Signalerne er i stereo (2 kanaler / kolonner).
Hvis vi har et system med to højttalere, giver det mening at betragte
kanalerne separat.
Altså vi ser kanalerne i forlængelse, som en mono serie med $M=2N$
samples. Denne løsning bruges, fordi det er sådan et menneske med to ører
og sæt hovedtelefoner ville opleve signalet :-) \\\\
En sum eller et gennemsnit på tværs af kanalerne ville betyde, at kanaler
ude af fase kunne eliminere hinanden.
Dette ville give mening som en simpel konvertering til mono, dvs. vi
kunne beregne mål på hvad der ville ske i et simpelt mono-system.\\\\
\textbf{Beregning:}~ Minimum og maksimum findes nemt med hhv. \texttt{min}~ og \texttt{max}.
I tidsdomænet er effekten af et signal proportionalt til kvadratet på
amplituden. For en sekvens $x(n) \in \mathbb{R}$, $n = 0,\ldots,N-1$
defineres $x_{pwr}(n) = |x(n)|^2 = x(n)^2 $.
I diskret tid er energien i signalet summen af ``effekterne'', dvs.
$E_x = \sum_{n=0}^{N-1} |x(n)|^2 $.
RMS-værdien kan så beregnes som kvadratroden af middeleffekten, dvs.
$x_{RMS} = \sqrt{\frac{1}{N}E_x}$.\\\\

\end{par} 

\begin{lstlisting}[language=Matlab, style=Matlab-editor]
signaler = {'s1'; 's2'; 's3'};
N = [length(s1); length(s2); length(s3)]; % antal diskrete tidsobservat.
M = 2*N;               % beregn samlet antal af datapunkter ~ målinger

s1_vec = reshape(s1,1,[]); %reshape matricer til søjlevektorer
s2_vec = reshape(s2,1,[]);
s3_vec = reshape(s3,1,[]);

minima = [min(s1_vec); min(s2_vec); min(s3_vec)];
maxima = [max(s1_vec); max(s2_vec); max(s3_vec)];
energi = [sum(s1_vec.^2); sum(s2_vec.^2); sum(s3_vec.^2)]; % kvadratsum
rms = [energi(1)/M(1); energi(2)/M(2); energi(3)/M(3)].^(1/2);

T = table(signaler, N, M, minima, maxima, energi, rms)
\end{lstlisting}

        \color{lightgray} \begin{verbatim}T =
  3×7 table
    signaler        N             M          minima     maxima       energi        rms   
    ________    __________    __________    ________    _______    __________    ________
      's1'       1.323e+06     2.646e+06     -1.0166     1.0191    2.5336e+05     0.30944
      's2'        3.36e+06      6.72e+06    -0.61796    0.62791         34641    0.071797
      's3'      1.4112e+06    2.8224e+06    -0.85016    0.76907        5662.2     0.04479
\end{verbatim} \color{black}
    \begin{par}

\begin{table}[H]
\centering
\begin{tabular}{| c | c | c | c | c |} \hline
Signal & Min (1;2) & Max (1;2) & Energi & RMS \\ \hline
$s_1$ & 0; 0 & 0; 0 & 0 & 0 \\ \hline
$s_2$ & 0; 0 & 0; 0 & 0 & 0 \\ \hline
$s_3$ & 0; 0 & 0; 0 & 0 & 0 \\ \hline
\end{tabular}\caption{Statistik på signalerne}\label{tab:stat}\end{table}

\end{par} 



\section{Venstre vs. højre kanal (for $s_1$)}

        


\section{Nedsampling af signal (for $s_1$)}

        


\section{Fade-out med envelopes (for $s_2$)}

        \begin{par}
\chapter{Konklusion}
\end{par} 



\end{document}

