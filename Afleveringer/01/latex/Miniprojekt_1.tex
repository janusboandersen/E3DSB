

\documentclass[a4paper]{report}
\usepackage[margin=1in]{geometry}
\usepackage{graphicx}
\usepackage{color}
\usepackage{amsfonts, amsmath, amssymb}
\usepackage{float} %ensures that figures / tables can be floated on page
\usepackage{hyperref}
\usepackage{polyglossia}
\setdefaultlanguage{danish}
\usepackage[binary-units=true]{siunitx}
\sisetup{detect-all}


\sloppy
\definecolor{lightgray}{gray}{0.5}
\setlength{\parindent}{0pt}

\usepackage{titlesec}
\titleformat{\chapter}{\huge\bf}{\thechapter.}{20pt}{\huge\bf}
\titleclass{\chapter}{straight}

\author{Janus Bo Andersen \thanks{ja67494@post.au.dk}}

\begin{document}
\pagenumbering{roman}



    
    
    \title{E3DSB miniprojekt 1 - Tidsdomæneanalyse}
        

    \maketitle

\tableofcontents
\newpage

\pagenumbering{arabic}

    \begin{par}

\chapter{Indledning}
Dette første miniprojekt i E3DSB behandler tre lydsignaler med analyser i tidsdomænet.
Opgaven er løst individuelt.
Matlab-koden kan findes på \url{https://github.com/janusboandersen/E3DSB}.

\end{par} \vspace{1em}
\begin{par}

\chapter{Analyse}

\end{par} \vspace{1em}



\section{De tre signaler}

\begin{par}
Noget tekst
\end{par} \vspace{1em}



\end{document}

